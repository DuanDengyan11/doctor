\iffalse
  % 本块代码被上方的 iffalse 注释掉,如需使用,请改为 iftrue
  % 使用 Noto 字体替换中文宋体、黑体
  \setCJKfamilyfont{\CJKrmdefault}[BoldFont=Noto Serif CJK SC Bold]{Noto Serif CJK SC}
  \renewcommand\songti{\CJKfamily{\CJKrmdefault}}
  \setCJKfamilyfont{\CJKsfdefault}[BoldFont=Noto Sans CJK SC Bold]{Noto Sans CJK SC Medium}
  \renewcommand\heiti{\CJKfamily{\CJKsfdefault}}
\fi

\iffalse
  % 本块代码被上方的 iffalse 注释掉,如需使用,请改为 iftrue
  % 在 XeLaTeX + ctexbook 环境下使用 Noto 日文字体
  \setCJKfamilyfont{mc}[BoldFont=Noto Serif CJK JP Bold]{Noto Serif CJK JP}
  \newcommand\mcfamily{\CJKfamily{mc}}
  \setCJKfamilyfont{gt}[BoldFont=Noto Sans CJK JP Bold]{Noto Sans CJK JP}
  \newcommand\gtfamily{\CJKfamily{gt}}
\fi


% 设置基本文档信息,\linebreak 前面不要有空格,否则在无需换行的场合,中文之间的空格无法消除
\nuaaset{
  thesisid = {},   % 论文编号
  title = {直升机协同吊挂控制技术研究},
  % title={基于功率消耗和鲁棒自适应博弈控制的负载分配策略}
  author = {段登燕},
  college = {航空学院},
  advisers = {李建波\quad 研究员},
  % applydate = {二〇一八年六月}  % 默认当前日期
  %
  % 本科
  major = {\LaTeX{} 科学与技术},
  studentid = {131810299},
  classid = {应用技术},           % 班级的名称
  industrialadvisers = {Jack Ma}, % 企业导师,若无请删除或注释本行
  % 硕/博士
  majorsubject = {航空宇航科学与技术},
  researchfield = {飞行力学与控制},
  libraryclassid = {V221},       % 中图分类号
  subjectclassid = {082501},      % 学科分类号
}
\nuaasetEn{
  title = {Research on control techonology of multi-lift system with helicopters carrying load cooperatively},
  author = {Duan Dengyan},
  college = {College of Aerospace Engineering},
  majorsubject = {Aerospace Science and Technology},
  advisers = {Prof.Li Jianbo},
  degreefull = {Doctor of Philosophy},
  % applydate = {June, 8012}
}

% 摘要
\begin{abstract}
本文介绍如何使用\nuaathesis{} 文档类撰写南京航空航天大学学位论文。

首先介绍如何获取并编译本文档,然后展示论文部件的实例,最后列举部分常用宏包的使用方法。
\end{abstract}
\keywords{学位论文, 模板, \nuaathesis}

\begin{abstractEn}
This document introduces \nuaathesis, the \LaTeX{} document class for NUAA Thesis.

First, we show how to get the source code and compile this document.
Then we provide snippets for figures, tables, equations, etc.
Finally we enforce some usage patterns.
\end{abstractEn}
\keywordsEn{NUAA thesis, document class, space is accepted here}


% 请按自己的论文排版需求,随意修改以下全局设置
\usepackage{cases}
\usepackage{hyperref}
\usepackage{color, soul}
\usepackage[utf8]{inputenc}
\usepackage{subfig}
\usepackage{rotating}
\usepackage[usenames,dvipsnames]{xcolor}
\usepackage{tikz}
\usetikzlibrary{decorations.pathreplacing, calligraphy}
\usetikzlibrary{quotes,angles}
\usetikzlibrary{shapes.geometric,arrows}
\usetikzlibrary{fit}
\usetikzlibrary{backgrounds}

\tikzstyle{arrow}     = [thick,->,>=stealth]
\tikzstyle{arrow_double}  = [thick,<->,>=stealth]
\tikzstyle{startstop} = [rectangle, rounded corners, minimum width=2.5cm, minimum height=1cm, text centered, draw=black, fill=green!30, align = center, font = \fontsize{10}{10}\selectfont]
\tikzstyle{process} = [rectangle, rounded corners, text centered, draw = black, minimum width = 2.5cm, minimum height = 1cm, align = center, font = \fontsize{10}{10}\selectfont]
\tikzstyle{process_red} = [rectangle, rounded corners, text centered, draw = black, minimum width = 2.5cm, minimum height = 1cm, fill = red!30, align = center, font = \fontsize{10}{10}\selectfont]
\tikzstyle{process_orange} = [rectangle, rounded corners, text centered, draw = black, minimum width = 2.5cm, minimum height = 1cm, fill = orange!30, align = center, font = \fontsize{10}{10}\selectfont]
\tikzstyle{process_blue} = [rectangle, rounded corners, text centered, draw = black, minimum width = 2.5cm, minimum height = 1cm, fill = blue!30, align = center, font = \fontsize{10}{10}\selectfont]
\tikzstyle{process_green} = [rectangle, rounded corners, text centered, draw = black, minimum width = 2.5cm, minimum height = 1cm, fill = green!30, align = center, font = \fontsize{10}{10}\selectfont]
\tikzstyle{process_purple} = [rectangle, rounded corners, text centered, draw = black, minimum width = 2.5cm, minimum height = 1cm, fill = purple!20, align = center, font = \fontsize{10}{10}\selectfont]
\tikzstyle{decision}  = [diamond,shape aspect=2.5, minimum width=3cm, minimum height=1cm, inner xsep=0,text centered, draw=black, fill=red!30, align = center, font = \fontsize{10}{10}\selectfont]
\tikzstyle{text_me} = [font = \fontsize{10}{10}\selectfont]

\tikzset{global scale/.style={
    scale=#1,
    every node/.append style={scale=#1}
  }
}

\usepackage{pgfplots}
\pgfplotsset{compat=1.16}
\pgfplotsset{
  table/search path={./fig/},
}
\usepackage{ifthen}
\usepackage{longtable}
\usepackage{siunitx}
\usepackage{listings}
\usepackage{multirow}
\usepackage{pifont}
\usepackage{gensymb}
\usepackage{bm}

\lstdefinestyle{lstStyleBase}{%
  basicstyle=\small\ttfamily,
  aboveskip=\medskipamount,
  belowskip=\medskipamount,
  lineskip=0pt,
  boxpos=c,
  showlines=false,
  extendedchars=true,
  upquote=true,
  tabsize=2,
  showtabs=false,
  showspaces=false,
  showstringspaces=false,
  numbers=left,
  numberstyle=\footnotesize,
  linewidth=\linewidth,
  xleftmargin=\parindent,
  xrightmargin=0pt,
  resetmargins=false,
  breaklines=true,
  breakatwhitespace=false,
  breakindent=0pt,
  breakautoindent=true,
  columns=flexible,
  keepspaces=true,
  framesep=3pt,
  rulesep=2pt,
  framerule=1pt,
  backgroundcolor=\color{red!10},
  stringstyle=\color{green!40!black!100},
  keywordstyle=\bfseries\color{blue!50!black},
  commentstyle=\slshape\color{black!60}}



%\usetikzlibrary{external}
%\tikzexternalize % activate!

\newcommand\cs[1]{\texttt{\textbackslash#1}}
\newcommand\pkg[1]{\texttt{#1}\textsuperscript{PKG}}
\newcommand\env[1]{\texttt{#1}}

\theoremstyle{nuaaplain}
\nuaatheoremchapu{definition}{定义}
\nuaatheoremchapu{assumption}{假设}
\nuaatheoremchap{exercise}{练习}
\nuaatheoremchap{nonsense}{胡诌}
\nuaatheoremg[句]{lines}{句子}
